\documentclass{beamer}
\usepackage[MeX]{polski}
\usepackage[utf8]{inputenc}
\usepackage{amsfonts}
\usepackage{graphicx}

%opening
\title{Masaż Bambusem}
\author{Krystian Kozłowski}
\date{\today}

\begin{document}
\frame{\titlepage}
	
	






\section{Opis}
\begin{frame}{Opis}
Głównym narzędziem są pałeczki bambusowe. Techniki manualne są wykorzystywane do rozpoczęcia, zakończenia oraz łączenia partii masażu pałeczkami. Najlepsze efekty daje z połaczeniem aromaterapią i muzyka relaksacyjną.
\end{frame}
\begin{frame}{Zdjęcie}
\begin{figure}
\centering
\includegraphics[width=3cm,]{pic/bambus.jpg}
\caption{Bambus oldhamii}

\end{figure}

\end{frame}
\section{Zalecenia}
\begin {frame}{Zalecenia}
Wskazania do masażu:
\begin{itemize}
\item przemęczenie i stres
\pause
\item brak odpowiedniego dotlenienia organizmu
\pause
\item bóle głowy, pleców
\pause
\item bóle mięśni
\pause
\item zły stan skóry
\end{itemize}

\end{frame}
\begin{thebibliography}{99}
\bibitem{Joa} Dylewska-Grzelakowska, Joanna: \textit{Kosmetyka stosowana}
Warszawa 1994.
\end{thebibliography}


\end{document}