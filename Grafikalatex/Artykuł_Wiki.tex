\documentclass[a4paper,12pt]{article}
\usepackage[MeX]{polski}
\usepackage[utf8]{inputenc}
\usepackage{graphicx}


%opening
\title{Ibson Barreto da Silva}
\author{Krystian Kozłowski}

\begin{document}

\maketitle

\begin{abstract}
Ibson Barreto da Silva znany jako Ibson (ur. 7 listopada 1983 w Niterói) – piłkarz brazylijski grający na pozycji środkowego pomocnika. Mierzy 177 cm wzrostu, waży 74 kg. Od 2014 roku jest zawodnikiem klubu Bologna FC.
\end{abstract}
\begin{center}
\includegraphics[scale=0.75]{pic/meh.jpg}
\end{center}


\section{Kariera klubowa}
Karierę piłkarską ,,Ibson'' rozpoczął w klubie Flamengo z miasta Rio de Janeiro. Do 2002 roku występował w drużynach juniorskich,a w 2003 roku awansował do pierwszej drużyny. 15 czerwca 2003 zadebiutował w barwach Flamengo w Série A w wygranym 2:1 meczu z Vasco da Gama. W debiutanckim sezonie rozegrał we Flamengo 7 spotkań, ale już w następnym był podstawowym zawodnikiem zespołu. Wraz z Flamengo dotarł do finału Copa do Brasil oraz wygrał dwa inne trofea - Campeonato Carioca i Taça Guanabara.

Na początku 2005 roku ,,Ibson'' przeszedł za 1,8 miliona euro z Flamengo do FC Porto[1]. W portugalskiej ekstraklasie zadebiutował 5 lutego 2005 w wygranym 2:1 wyjazdowym spotkaniu z GD Estoril-Praia, gdy w ostatniej minucie zmienił Héldera Postigę. Następnie do końca sezonu grał w podstawowym składzie Porto i zdobył w nim zarówno Puchar Portugalii, Superpuchar oraz mistrzostwo kraju. W sezonie 2005/2006 i 2006/2007 był rezerwowym w Porto. Wówczas dwukrotnie wywalczył mistrzostwo kraju oraz po jednym razie puchar i superpuchar kraju (oba w 2006 roku).

Na początku 2007 roku ,,Ibson'' został wypożyczony z Porto do Flamengo. W 2007 roku strzelił 6 goli dla Flamengo, a w następnym - 11. W lipcu 2008 wypożyczenie zostało przedłużone o kolejny rok[2]. W 2008 i 2009 roku wywalczył z Flamengo mistrzostwo stanowe, a w tym drugim przypadku został też mistrzem kraju.

13 lipca 2009 roku ,,Ibson'' został sprzedany z Porto do Spartaka Moskwa za kwotę 5 milionów euro[3]. Zadebiutował w nim 1 sierpnia 2009 w wygranym 4:0 domowym spotkaniu z Kubaniem Krasnodar.'

W 2011 roku ,,Ibson'' odszedł ze Spartaka do Santosu.

\section{Statystyki}


\begin{table}[b]
\begin{tabular}{lccccr}
\hline
\textbf{Sezon}&\textbf{Klub}&\textbf{Kraj}&\textbf{Liga}&\textbf{Mecze}&\textbf{Bramki}\\
\hline
2003&Flamengo&Brazylia&1 Liga&7&0 \\
2004&Flamengo&Brazylia&1 Liga&42&6 \\
2004/2005&FC Porto&Portugalia&1 Liga&15&1 \\
2005/2006&FC Porto&Portugalia&1 Liga&18&1 \\
2006/2007&FC Porto&Portugalia&1 Liga&13&0 \\
2007&Flamengo&Brazylia&1 Liga&22&6 \\
2008&Flamengo&Brazylia&1 Liga&33&11 \\
2009&Flamengo&Brazylia&1 Liga&9&0 \\
2009&Spartak Mos.&Rosja&1 Liga&6&0 \\
20010&Spartak Mos.&Rosja&1 Liga&28&2 \\
2011/12&Spartak Mos.&Rosja&1 Liga&10&1 \\
2011&Santos FC&Brazylia&1 Liga&19&0 \\
2011&Flamengo&Brazylia&1 Liga&33&1 \\
2011&Corinthians&Brazylia&1 Liga&20&0 \\
\hline
\end{tabular}
\caption{Statystyki}
\label{tabela1}
\end{table}


Tabela wynków zawodnika \ref{tabela1} pokazuje, że nie był on wybitnym piłkarzem.
\end{document}